The 2013 paper written by Ari Juels and Ronald L. Rivest, found  there%[here](http://people.csail.mit.edu/rivest/honeywords/paper.pdf), 
introduces a fascinating way of detecting offline brute-force password-cracking. In a similar style to honeypots, honeywords are stored with the user's password. Should the username/password files be compromised, an adversary is presented with honeywords in addition to a user's genuine password. If this false password is entered into the authentication system with the username, a flag is raised indicating that someone has compromised the files and is attempting to masquerade as the user. \\

Despite being a clever scheme, several problems still remain before an implementation can be successfully deployed. The most pertinent of these problems is the honeywords generation. Given a password, a set of honeywords must be generated in a manner such that the adversary will have the most difficulty in guessing the genuine password from the complete set. Modern-UI, which forces users to adhere to a more strict standard of creating passwords as per the paper, successfully eliminates the need for this algorithm. However, the logistics and stigma behind forcing users to adhere to a particular pattern of passwords can be infeasable or undesirable. Therefore, this algorithm will attempt to accept *any* given password and produce a set of corresponding honeywords designed to appear as feasible passwords.\\


%% comment
% 2-3 sentences on honeywords, chaffing. The strategies discussed in prior literature leave much room for improvement. Enter HoneyBabel. Chaffing strategies we propose (2-3 sentences). Sentence on our analysis/discussion/conclusion. Sentence on our contribution to this research area.


%%% Local Variables: 
%%% mode: latex
%%% TeX-master: "mics_paper"
%%% End: 

Recent research has explored the detection of offline brute-force password-cracking using a mechanism known as honeywords. Honeywords are dummy passwords that are stored alongside users' passwords. If the password file is compromised and a honeyword is subsequently used for authentication, a flag is raised indicating a potential masquerading attempt. Several issues must be addressed for a successful deployment of a honeywords system, the most pertinent being honeyword generation. Given a password, a set of honeywords must be generated such that the adversary will have an increased difficulty for guessing the genuine password from the complete set. To this end, we have created an algorithm that will accept an arbitrary password and produce a set of corresponding honeywords designed to appear as user-chosen passwords. In this paper, we discuss this algorithm and its potential for improving the detection of offline password-cracking attacks.\\

%%% Local Variables: 
%%% mode: latex
%%% TeX-master: "mics_paper"
%%% End: 
